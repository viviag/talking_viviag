\documentclass[english,12pt]{article}
\usepackage[utf8]{inputenc}
\usepackage[T2A]{fontenc}
\usepackage{amsmath,amssymb,amsthm}
\usepackage[a4paper,hmargin=2.5cm,vmargin=2.5cm]{geometry}
\usepackage[english]{babel}
\usepackage{tikz-cd}
\usepackage{enumitem}

\newcounter{stmcounter}[section]
\newcounter{thcounter}

\numberwithin{equation}{section}
\renewcommand{\thestmcounter}{\thesection.\arabic{stmcounter}}

\newtheorem{proposition}[stmcounter]{Утверждение}
\newtheorem{lemma}[stmcounter]{Лемма}
\newtheorem{corollary}[stmcounter]{Следствие}
\newtheorem{statement}[stmcounter]{Утверждение}
\newtheorem{theorem}[thcounter]{Теорема}

\theoremstyle{definition}
\newtheorem{definition}[stmcounter]{Определение}
\newtheorem{property}[stmcounter]{Свойство}

\theoremstyle{remark}
\newtheorem{remark}[stmcounter]{Замечание}
\newtheorem{example}[stmcounter]{Пример}

% \newenvironment{pf}{\noindent\textbf{Доказательство.} ~ \par}{\qed}
\newenvironment{pf}{\noindent\textbf{Доказательство.}}{\qed}
\newcommand{\define}[1]{{\textit{#1}}}

\renewcommand{\leq}{\leqslant}
\renewcommand{\geq}{\geqslant}

\begin{document}

\section{Опорный конспект доклада}

\subsection{Предыдущая часть}

Некоторая формула: $a^2$\\

Общая постановка задачи --- есть покрытие $\mathcal{U}$ некоторого топологического подмногообразия с краем $\mathcal{D}$ в $\mathbb{R}^n$ с шарами радиуса $r_c$ с центрами в точках множества $\mathrm{X}$. Хотим достаточное условие того, что связное множество $\mathcal{D}_r = \mathcal{D} - N_r$, где $N_r = \{x \in \mathcal{D}:\; d(x,\delta D) \leq r\}$ и $r$ такое, что $\mathcal{D}_r$ непусто, лежит в $\mathcal{U}$.\\

Мы хотим обойтись минимальными техническими возможностями узлов. Предполагается, что они могут посылать какие-то сигналы, но по ним невычислимы точные расстояния и они не знают карту покрытия. Естественные на практике требования.\\

Введём также $\Sigma = \{x \in \mathcal{D}:\; d(x,\delta D) = r\}$ --- внутреннюю границу $N_r$.

\subsection{Первая петля}

Заметим, что в описанных условиях $\mathcal{D}$ --- гладкое ориентированное многообразие с краем.

\begin{proposition}
  Гомология пары $H_d(\mathcal{U} \cup N_r, N_r)$ нетривиальна тогда и только тогда, когда $\mathcal{D} - N_r \subset \mathcal{U}$.
\end{proposition}

\begin{pf}
  По аксиоме вырезания $H_d(\mathcal{D}, N_r) = H_d(\mathcal{D}_r \cup \Sigma, \Sigma)$ По двойственности Пуанкаре-Лефшеца для многообразий с краем $H_d(\mathcal{D}_r \cup \Sigma, \Sigma) = H^0(\mathcal{D}_r \cup \Sigma) = \mathbb{Z}$. Последнее следует из связности $\mathcal{D}_r$.\\

  Пусть $\mathcal{D} - N_r \subset \mathcal{U}$. Тогда $H_d(\mathcal{U} \cup N_r, N_r)$ содержит порождающий цикл $H_d(\mathcal{D}, N_r)$, следовательно, нетривиальна.\\

  Рассмотрим расширенную приграничную область $\overline{N_r} = \mathbb{R}^d - (\mathcal{D} - N_r)$. Пусть $Err = (\mathcal{D} - N_r) - \mathcal{U}$ По вырезанию $H_d(\mathbb{R}^d - Err, \overline{N_r}) = H_d(\mathcal{U} \cup N_r, N_r)$.

  По двойственности Александера $H_d(\mathbb{R}^d - Err, \overline{N_r}) = H^0(\mathcal{D}_r, (\mathcal{D} - N_r) - \mathcal{U}) = 0$.
\end{pf}

Это могло бы быть хорошим критерием, если бы нам были доступны нужные гомологии напрямую, это неверно. Мы хотим к нему приблизиться.

\subsection{Вторая петля}

Мы везде рассматриваем гомологии относительно края. Вычислимый инвариант, который хочется построить, тоже должен будет учитывать край.\\

Вносим в параметры узлов умение детектировать наличие границы в радиусе $r_f$. При $r$ достаточно превышающем $r_f$ гомологии $H_d(Cech(\mathcal{D}), Cech(N_r))$ комплекса и подкомплекса Чеха, построенного по всем узлам и узлам близко к границе соответственно, будут равны искомым (теорема о нерве). Но для вычисления комплекса Чеха нужны точные данные о попарных расстояниях, которых нет.

\subsection{Третья петля}

Данных хватает, чтобы построить (абстрактный) комплекс Вьеториса-Рипса с радиусом $r_c$. Как он связан с топологией пространства, непонятно совершенно. Здесь рисунок с октаэдром и точками на окружности, первый из двух в статье. Зато можно сказать, что узлы посылают свои идентификаторы и в некотором радиусе $r_s < r_c$ другие узлы их видят. И строить комплекс Рипса радиуса $r_s$. $r_s$ можно задать такой, чтобы $Rips_{r_s} \subset Cech_{r_c} \subset Rips_{r_c}$. Это теорема 2.5 статьи. Тонкость, что в коплексе Чеха $r_c$ --- диаметр.

Далее разбор теоремы 2.5, схема:
\begin{enumerate}
  \item Правое включение по определению.
  \item Левое переформулируется как наличие общей точки у шаров в нужном комплексе Рипса.
  \item Для $d+1$ и меньших наборов см. выкладки.
  \item Применяем теорему Хелли.
\end{enumerate}

Выкладки (предположение, что точки на попарном расстоянии не выше $\varepsilon$):\\

Доказываем для множества $d'+1$ точек, где $d' < d$.\\

Рассмотрим $f(y) = max(d(x_0, y))$, у этой функции есть глобальный минимум $y_0$ и соответствующие ему критические точки $\{x_0,\ldots,x_{d''}\}$. Предположим, что есть вектор $v$, разделяющий выпуклую оболочку критических точек от $y_0$ (то есть для всех критических точек $(x_i-y_0, v) > 0$). Тогда для любой критической точки $x$ $(x-y_0, x-y_0) = (x-(y_0 + \lambda v) + \lambda v, x-(y_0 + \lambda v) + \lambda v) = (x-(y_0 + \lambda v), x-(y_0 + \lambda v)) + 2(x-(y_0 + \lambda v), \lambda v) + (\lambda v, \lambda v)$.\\

$2(x-(y_0 + \lambda v), \lambda v) = 2\lambda(x-y_0,v) - 2\lambda^2(v,v)$; $(x-y_0, x-y_0) = (x-(y_0 + \lambda v), x-(y_0 + \lambda v)) + 2\lambda(x-y_0,v) - (\lambda v, \lambda v)$. Для достаточно маленьких лямбд получаем, что $d(x,y_0 + \lambda v) < d(x,y_0)$, что противоречит минимальности. Следовательно, $y_0$ лежит в выпуклой оболочке критических точек $f$.\\

Значит, существует выпуклая комбинация $y_0 = a_0x_0 + \ldots + a_{d''}x_{d''}$, без ограничения общности $a_0$ --- наибольший коэффициент. Сдвинули на $y_0$: $0 = a_0x'_0 + \ldots + a_{d''}x'_{d''}$, выразили $x'_0$ через вот это всё, скалярно домножили на $x'_0$ с обеих сторон. Получили $-f(y_0)^2 = -(x'_0,x'_0) = \sum_{i=1}^{d''}\frac{a_i}{a_0}(x_i,x_0))$.\\

Для какого-то $i$ справа верно, что $\frac{a_i}{a_0}(x_i,x_0)) \leq -\frac{(x'_0,x'_0)}{d''}$, что можно ослабить как $\frac{f(y_0)^2}{d} \leq -(x'_0,x'_i)$. При этом $f(y_0)^2 = (x'_0,x'_0) = (x'_i,x'_i)$.\\

Суммируем: $f(y_0)^2(1+\frac{2}{d}+1) \leq (x'_0,x'_0) - 2(x'_0,x'_i) + (x'_i,x'_i) = (x'_0-x'_i, x'_0-x'_i) = (x_0-x_i, x_0-x_i) \leq \varepsilon^2$.\\
Получаем $f(y_0) \leq \frac{\varepsilon}{2}\sqrt{\frac{2d}{d+1}}$. Значит, шары радиуса правой части с центрами в точках набора встретятся в $y_0$.\\

Получили ограничение (оптимальное --- достигается уже нарисованным примером) на $r_s$.
Допишем условие на сенсорную сеть: $r_c \geq r_s\sqrt{\frac{d}{2(d+1)}}$.

Некоторый набор следствий или сходных утверждений:

\begin{proposition}
  Если попарные расстояния набора точек $X = \{x_0, \ldots, x_k\}$ не превосходят $\delta$, расстояние от любой точки их выпуклой оболочки до какой-то из них не превосходит $\epsilon = \delta\sqrt{\frac{d}{2(d+1)}}$.
\end{proposition}

\begin{pf}
  Запишем уравнение для $p$ как для точки выпуклой оболочки, сдвинем его на $p$, скалярно перемножим с общей точкой шаров из коплекса Чеха на $X$ радиуса $\epsilon$.\\

  Среди слагаемых справа есть хотя бы одно неположительное, т.е. $(x'_i,y') \leq 0$. Тогда $\epsilon \geq (x_i-y,x_i-y) = (x'_i-y',x'_i-y') = (x'_i,x'_i) -2(x'_i,y') + (y',y') \geq (x'_i,x'_i) = (x-p,x-p)$.
\end{pf}

В этой теореме можно взять $k = d$, потому что рассуждение помещается в $k$-мерное подпространство.\\

\begin{corollary}
  При имеющейся связи на $r_s$ и $r_c$ геометрическая реализация любого симплекса $R_s$ на своих вершинах лежит в $U$.
\end{corollary}

Напоминание, что геометрический симплекс --- выпуклая оболочка его вершин.\\

Строим комплексы Рипса для радиуса $r_s$ полный и приграничный $\mathcal{F}_{r_s}$ --- на вершинах, видящих границу. В этот раз надежда, что попадём во что-то правдоподобное, заявляя нетривиальность $H_d(Rips_{r_s}(\mathcal{D}), \mathcal{F}_{r_s})$. Здесь контрпример с приграничным циклом и круговой областью как опровержение, это второй рисунок статьи (стоит заметить, что все грани с красной вершиной затянуты). Его можно увидеть, смотря на точную последовательность пары, в которой возникает изоморфизм $H_1(\mathcal{F}_{r_s}) = H_2(Rips_{r_s},\mathcal{F}_{r_s})$ --- весь комплекс является конусом над циклом в $\mathcal{F}_{r_s}$, следовательно, стягиваем.\\

\subsection{Четвёртая петля}

Заметим, что если узлы посылают два типа сигналов с разными радиусами охвата, одна пара комплексов Рипса вкладывается в другую и в этой второй при достаточно большом различии радиусов дыры на границе затягиваются. Вводим радиус приёма второго типа сигнала $r_w > r_s$.\\

Признак, который мы доказываем, формулируется для данного многообразия следующим образом: индуцированное вложением отображение старших относительных гомологий $H_d(Rips_{r_s},\mathcal{F}_{r_s}) \xrightarrow{i^{\star}} H_d(Rips_{r_w},\mathcal{F}_{r_w})$ нетривиально.\\

Заметим, что для случая $\mathcal{D}_r$, лежащей в каком-то симплексе $Rips_{r_s}$, теорема верна в силу следствия 1.3. Заметим также, что теорема должна содержать условие на связь $r_w$ и $r_s$, более сильное, нежели их неравенство.\\


Следующее утверждение верно при дополнительном условии на $r$.
\begin{proposition}
  Либо геометрическая реализация любого симплекса $\mathcal{F}_{r_s}$ лежит в $\overline{N_r}$, либо $\mathcal{D}_r \subset \mathcal{U}$.
\end{proposition}

\begin{pf}
  По теореме Каратеодори любая точка выпуклой оболочки множества точек в $\mathbb{R}^d$ лежит в каком-то $d$-симплексе. То есть достаточно проверить $d$-мерный остов $\mathbb{F}_{r_s}$.\\

  Для любого геометрического симплекса на узлах размерности не больше, чем $d-1$, верно, что любая его точка отстоит от какой-то из его вершин не более чем на $r_s\sqrt{\frac{d-1}{2d}}$. Вершины лежат на расстоянии не больше $r_f$ от границы, следовательно, мы получили по неравенству наименьшее возможное значение $r = r_s\sqrt{\frac{d-1}{2d}} + r_f$.\\

  Рассуждение для $d$-мерного остова можно провести так же, утолщив границу, но этого можно не делать. Пусть $\sigma$ --- $d$-симплекс. Граница этого симплекса лежит в $d-1$-мерном остове, следовательно, в $\overline{N_r}$. Следовательно, или он полностью содержит $D_r$ (см. первый рисунок), либо он в $\overline{N_r}$.
\end{pf}

Утверждается, что это оптимальное значение $r$, тут должен быть пример.\\

Теперь начнём доказывать признак.\\

Сначала применим геометрическую реализацию, которая в силу предыдущего утверждения отправит $(Rips_{r_s}, \mathcal{F}_{r_s})$ в $(\mathbb{R}^d,\overline{N_r})$, и запишем две точные последовательности пары, связанные гомоморфизмом $\sigma$ в коммутативную диаграму.\\

$\ldots \to H_d(Rips_{r_s}, \mathcal{F}_{r_s}) \xrightarrow{\delta_{\star}} H_{d-1}(\mathcal{F}_{r_s}) \to \ldots$\\
$\ldots \to H_d(\mathbb{R}^d,\overline{N_r}) \xrightarrow{\delta_{\star}} H_{d-1}(\overline{N_r}) \to \ldots$.\\

Рассмотрим класс $[\alpha]$, образ которого под действием $i_{\star}$ нетривиален. Верно, что $\sigma_{star} \circ \delta_{\star}[\alpha]$ или равен нулю, или не равен. Пусть не равен.\\

В силу коммутативности диаграммы $\sigma_{\star}[\alpha] \neq 0$. Пусть $\mathcal{U}$ не содержит $\mathcal{D}_r$, тогда рассмотрим $Err$. Поскольку $\sigma(Rips_{r_s})$ лежит в $\mathcal{U}$, $\sigma$ пропускается через $H_d(\mathbb{R}^d - Err, \overline{N_r}) = 0$. Это уже знакомое рассуждение. Пришли к противоречию с $\sigma([\alpha]) \neq 0$.\\

Следовательно, случай, который должен зависеть от $r_w$ --- случай $\sigma_{\star} \circ \delta_{\star}[\alpha] = 0$. Мы ищем условие на $r_w$ и $r_s$ такое, чтобы этот случай был невозможен.

\subsection{Финал}

Выберем геометрический относительный цикл $\alpha$, представляющий $[\alpha]$. Его геометрическая граница лежит в $\mathcal{F}_{r_s}$. Зададим функцию знакового расстояния $h(y)$, положительную снаружи от $\Sigma$. На симплексе $\sigma \subset \partial \alpha$ она положительна. При этом $\sigma$ --- граница симплекса $\tau$ из $\alpha$, то есть из $Rips_{r_s} - \mathcal{F}_{r_s}$. Для другой вершины $\tau$ $y$ $h(y) < 0$.\\

Пусть $p$ --- внутренняя точка $\tau$. По неравенству треугольника $h(p) \leq h(y) + d(p,y) < 0 + r_s = r_s$. $p$ отстоит от какой-то вершины $x \in \tau$ на $r_s\sqrt{\frac{d-1}{2d}}$, имеем $h(p) \geq h(x) - d(p,x) \geq 0 - r_s\sqrt{\frac{d-1}{2d}}$. Тем самым мы запихнули любой симплекс границы относительного цикла в полосу $S$ около $\Sigma$.\\

Теперь нам нужно некоторое утверждение про $S$.
\begin{proposition}
  Пусть $S$ гомеоморфна $d-1$-многообразию, умноженному на отрезок, и накрывается отрезками длины не больше $\Delta$. Пусть $X$ --- набор точек, формирующих цикл в $S$, $[\gamma] \in H_{d-1}(R_{\epsilon}(X))$ и $\gamma$ целиком лежит в $S$.\\

  Тогда $[y] \in H_{d-1}(S) = 0$ влечёт $[y] \in H_{d-1}(Rips_{\varepsilon}(X))$, где $\varepsilon = \sqrt{\Delta^2 + 2\epsilon^2\frac{d-1}{d}}$.
\end{proposition}

\begin{pf}
  Рисунок, на нём теорема Пифагора. Итог --- утверждение, что покрытие $U$ шарами радиуса $\frac{\Delta}{2}$ с центрами по всех точках геометрического цикла лежит в покрытии $U'$ радиуса $\frac{\varepsilon}{2}$ с центрами в вершинах.\\

  Пусть $[\gamma]$ нетривиален в $H_{d-1}(Rips_{\varepsilon}(X))$, но тривиален в $H_{d-1}(S)$. Цикл $[\gamma]$ нетривиален в комплексе Чеха для $U'$, через гомологии которого пропускается индуцированный гомоморфизм из меньшего Рипса в больший. Существует точка $p \in S - U'$, вокруг которой обходит $\gamma$ --- утверждается, что по двойственности Александера, видимо, имеется ввиду аргумент из начала доклада для $H_{d-1}(Rips_{\varepsilon},U)$ и $H_{d-1}(S,U)$, но как-то хитро.\\

  Через эту точку проходит отрезок длины не больше $\delta$, соединяющий концы, он пересекает $\gamma$ в двух точках в двух сторон от $p$, получаем, что этот отрезок лежит в объединении шаров радиуса $\frac{\Delta}{2}$, что противоречит $p \not\in U'$.
\end{pf}

  Потребуем от полосы $S$ вдоль $\Sigma$, чтобы она удовлетворяла условиям утверждения. В предпосылках это записывается как требование на радиусы инъективности $\Sigma$ во внешность и во внутренность относительно $\mathcal{D}_r$, но в ремарках уточняется, что это требование можно ослабить и гладкость полосы не требовать. По предположению случая $\partial \alpha$ гомологичен нулю в $S$. Берём $\Delta = r_s(1+\sqrt{\frac{d-1}{2d}})$, получаем, что класс $\partial \alpha$ нулевой в комплексе Рипса радиуса $r_m = \sqrt{\frac{7d-5+2\sqrt{2d(d-1)}}{2d}}$.\\

  Диаграммным поиском по диаграмме из девяти элементов из конца статьи находим элемент $H_d(Rips_{r_m})$. Гомологии $d$-мерного подмножества $\mathbb{R}^d$ нулевые, следовательно, гомологии комплекса Чеха подходящего радиуса тоже. Наложим связь на $r_m$ и $r_w$, следующую из утверждения о двойном вложении, получим, что вложении $d$-x гомологий $R_{r_*}$ пропускается через 0. Из диаграммы получаем противоречие с посылкой теоремы.

\end{document}
