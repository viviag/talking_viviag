\documentclass[10pt]{beamer}
\usepackage[utf8]{inputenc}
\usepackage[T2A]{fontenc}
\usepackage[russian, english]{babel}

\babelfont{rm}{Droid Serif}
\babelfont{sf}{Droid Sans}

\usepackage{xeCJK}
\usepackage{graphicx}
\usepackage{mathtools}
\usepackage{tikz-cd}
\usepackage{utopia} %font utopia imported
\usetheme{Madrid}
\usecolortheme{dolphin}

\newcommand{\define}[1]{{\textit{#1}}}

% set colors
\definecolor{myNewColorA}{RGB}{25,25,112}
\definecolor{myNewColorB}{RGB}{25,25,112}
\definecolor{myNewColorC}{RGB}{25,25,112}
\setbeamercolor*{palette primary}{bg=myNewColorC}
\setbeamercolor*{palette secondary}{bg=myNewColorB, fg = white}
\setbeamercolor*{palette tertiary}{bg=myNewColorA, fg = white}
\setbeamercolor*{titlelike}{fg=myNewColorA}
\setbeamercolor*{title}{bg=myNewColorA, fg = white}
\setbeamercolor*{item}{fg=myNewColorA}
\setbeamercolor*{caption name}{fg=myNewColorA}
\usefonttheme{professionalfonts}
\usepackage{natbib}
\usepackage{hyperref}
%------------------------------------------------------------
\setbeamerfont{title}{size=\large}
\setbeamerfont{subtitle}{size=\small}
\setbeamerfont{author}{size=\small}
\setbeamerfont{date}{size=\small}
\setbeamerfont{institute}{size=\small}

%------------------------------------------------------------
%This block of commands puts the table of contents at the 
%beginning of each section and highlights the current section:
%\AtBeginSection[]
%{
%  \begin{frame}
%    \frametitle{Contents}
%    \tableofcontents[currentsection]
%  \end{frame}
%}
\AtBeginSection[]{
  \begin{frame}
  \vfill
  \centering
  \begin{beamercolorbox}[sep=8pt,center,shadow=true,rounded=true]{title}
    \usebeamerfont{title}\insertsectionhead\par%
  \end{beamercolorbox}
  \vfill
  \end{frame}
}
%------------------------------------------------------------

\begin{document}

%------------------------------------------------------------
\section{Идеалы колец $\mathbb{Z}_4[x]/(x^n-1)$ для $n = 2^e$}

\begin{frame}{На основе статьи}

\begin{block}{Cyclic codes of length $2^e$ over $\mathbb{Z}_4$}
  \url{https://core.ac.uk/download/pdf/82251738.pdf}\\
  \vspace{0.2cm}
  Авторы: Taher Abualrub, Robert Oehmke
\end{block}

\end{frame}

\begin{frame}{Идеалы в кольце $\mathbb{Z}_2[x]/(x^n-1)$}

 Главные, порождены делителями $x^n-1$.\\
 Разложим $x^n-1$: $x^{2^e}-1 = (x^{2^{e-1}}-1)^2 = (x-1)^n$.\\
 \pause
 \vspace{0.4cm}
 НОД всех делителей --- $x-1$. На другом языке сумма всех идеалов равна $(x-1)$.\\
 Cледовательно, $(x-1)$ --- единственный максимальный идеал.\\
 Он нильпотентен степени $n$. Следовательно, это кольцо Галуа.

\end{frame}

\begin{frame}{Идеалы при сюръективном отображении}

 Пусть $\phi : A \to B$ --- сюръективное отображение.\\
 По теореме об изоморфизме $A/\ker{\phi} \cong B$\\
 \vspace{0,4cm}
 Следовательно, (максимальные) идеалы $A$, содержащие $\ker{\phi}$, биективны (максимальным) идеалам $B$.

\end{frame}

\begin{frame}{Редукция по модулю $p$}

\begin{enumerate}
 \item $\phi : R = \mathbb{Z}_4[x]/(x^n-1) \to \mathbb{Z}_2[x]/(x^n-1)$.\\
 \item $\ker{\phi} = (2)$\\
 \item $\phi^{-1}((x-1)) = (2) + (x-1) = (2, x-1)$\\
 \item Это единственный максимальный идеал, нильпотентный.\\
 \item Является ли он главным?
\end{enumerate}

\end{frame}

\begin{frame}{Характеризация пересечения максимальных идеалов}

  \begin{block}{Утверждение (курс коммутативной алгебры)}
      Элемент $x$ принадлежит пересечению $J$ максимальных идеалов коммутативного кольца $A$ тогда и только тогда, когда $1-xy$ обратим для любого $y \in A$
  \end{block}
   \pause
   Если элемент $a = 1 - xy_0$ необратим, он содержится в максимальном идеале $m$. По предположению $x \in m$. Тогда $xy_0 \in m$ и $1 \in m$. Противоречие, доказали, что все элементы $J$ удовлетворяют этому свойству.\\
   \pause
   \vspace{0.4cm}
   В другую сторону. Пусть $m$ --- какой-то максимальный идеал и $x \not\in m$. Тогда $(x) + m = A$, в частности для каких-то $w \in m$ и $y_0$ $w + xy_0 = 1$. Следовательно, $1 - xy_0 \in m$, а значит, необратим.

   
\end{frame}

\begin{frame}{Максимальный идеал в локальном кольце}

\begin{block}{Утверждение}
  Пусть $(A, m)$ --- локальное кольцо и $m = (a)$. Пусть $m = (a_1,\ldots,a_n)$. Тогда $m=(a_i)$ для какого-то $i \in {1,\ldots,n}$.
\end{block}
\pause
 Для $n=1$ утверждение тавтологично. Пусть оно верно для $n=k$, рассмотрим $k+1$.\\
 \vspace{0.3cm}
 Верно, что $a_1 = ra$ (1). Если $r$ обратим, $m = (a_1)$. Иначе $r = \sum_1^{k+1}{\alpha_ia_i}$.\\
 Подставим в (1): $a_1(1-\alpha_1a) = \sum_2^{k+1}{a\alpha_ia_i}$. Множитель при $a_1$ обратим, поскольку максимальный идеал единственен, значит, $m = (a_2,\ldots,a_{k+1})$\\
 \pause
 \vspace{0.5cm}
 Заключаем, что идеал $(2,x-1) \subset R$ не является главным.
\end{frame}

\begin{frame}{Классификация идеалов.}
  Мы зафиксировали, что $R$ не является кольцом главных идеалов.\\
  \vspace{0.4cm}
  Пусть $I$ --- идеал и $g \in I$ --- элемент минимальной степени.\\
  \pause
  \vspace{0.4cm}
  Пусть $g$ --- унитарный многочлен. Тогда на него можно делить с остатком в $\mathbb{Z}_4[x]$, получая остаток меньшей степени.\\
  \vspace{0.4cm}
  Поделив все элементы $I$, представленные собой в $Z_4[x]$, получили, что $I = (g)$.
  Нулевой элемент можно представить как $x^n-1$, значит, $g \;|\; x^n-1$.\\
  \vspace{0.6cm}
  Это рассуждение дословно повторяет общее рассуждение, классифицирующее идеалы $\mathbb{Z}_2[x]/(x^n-1)$, на которое мы ссылались в начале.
  
\end{frame}

\begin{frame}{Отступление: разложение $x^n-1$ на множители}
    $x^4-1 = (x^2+1)(x-1)(x+1) = (x^2+2x-1)(x-1)(x-1) = (x^2+2x-1)(x+1)(x+1)$
\end{frame}

\begin{frame}{Классификация идеалов}
  В $R$ есть неглавные идеалы. Как минимум их элементы минимальной степени не унитарны.\\
  \vspace{0.4cm}
  Пусть $g$ не унитарен. То есть старший коэффициент -- двойка.\\
  Если $2g \neq 0$, степень $2g$ меньше степени $g$, значит, $g = 2q$ для какого-то многочлена $q$ с единичными коэффициентами.  
\end{frame}


\begin{frame}{Классификация идеалов}
  Пусть в $I$ нет унитарных многочленов.\\
  \vspace{0.3cm}
  Рассмотрим множество элементов $I$, которые не делятся на $g$. Если оно не пусто, в нём есть элемент $r$ минимальной степени $u$. Отметим, что $u \geq s  = \operatorname{deg}(g)$. Рассмотрим $w = r - 2qx^{u-s}$.\\
  \vspace{0.3cm}
  Он имеет степень, меньшую $r$, следовательно, делится на $2q$ (или нулевой, что подходит). Тогда и $r$ делится на $2q$. Противоречие.\\
  \vspace{0.5cm}
  Следовательно, в этом случае $I = (2q)$.\\
  Мы также получили утверждение, что все неглавные идеалы содержат унитарный многочлен.
\end{frame}

\begin{frame}{Классификация идеалов}
  Пусть $I$ содержит унитарный многочлен.\\
  \vspace{0.4cm}
  Множество унитарных многочленов содержит элемент $f$ минимальной степени $t$. Множество многочленов степени меньше $t$ попадает в условия предыдущего случая и все его элементы делятся на $2q$.\\
  \vspace{0.3cm}
  \pause
  Пусть $a \in I$ --- многочлен степени не меньше $t$.\\
  Поделим его с остатком на $f$: $a = a'f + r = af + 2qr'$.\\
  \vspace{0.2cm}
  Таким образом $I = (f) +  (2q) = (f,2q)$
\end{frame}

\begin{frame}{Классификация идеалов}
  \begin{block}{Теорема}
    Итого имеем идеалы следующих видов:\\
    \begin{enumerate}
      \item $(g)$, где $g$ --- делитель $x^n-1$
      \item $(2q)$, где $q$ имеет единичные коэффициенты
      \item $(f,2q)$, где $f$ унитарный.
    \end{enumerate}
  \end{block}
  \vspace{0.4cm}

  Условие на $f = f_1 + 2f_2$ можно усилить. Вычитая множители вида $2qx^k$, можно добиться того, чтобы степень $f_2$ была меньше $s$.\\
  \vspace{0.2cm}
  Образ идеала при сюръективном отображении --- идеал, а образующая отображается в образующую. Отсюда можно заключить также, что $\phi(f_1) \;|\; x^n-1$, $\phi(q) \;|\; x^n - 1$ и в силу неравенства на степени $\phi(q) \;|\; \phi(f_1)$.
\end{frame}

\begin{frame}{Классификация идеалов}
  Чего заключить нельзя:\\
  \vspace{0.4cm}
  Нельзя пользоваться подъёмом Гензеля и поднимать им разложение $x^n - 1$. К тому же их несколько.\\
  \vspace{0.4cm}
  Но даже если бы было можно, это бы не помогло описать все идеалы.\\
  \begin{example}
    Главный идеал $(x^3+x^2-x-1) \subset \mathbb{Z}_4[x]/(x^4-1)$ не порождается делителем $x^4 - 1$.
  \end{example}
    Достаточно поделить: $x^4 - 1 = (x-1)(x^3 + x^2 - x - 1) + 2(x^2 + 1)$.
\end{frame}

\end{document}



